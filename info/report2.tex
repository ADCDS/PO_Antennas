\documentclass{article}
\usepackage[utf8]{inputenc}

\title{Pesquisa Operacional - Relatório 08}
\author{Adriel Cardoso dos Santos}
\date{Maio 2018}
\usepackage{amsmath}
\usepackage{amssymb}
\usepackage[]{algorithm2e}
\begin{document}

    \maketitle

    \section{Descrição do problema}

    Foi pedido a comparação entre os dois modelos que solucionam diferentes problemas de distribuição de banda através de antenas.
    A primeira definição permite que o solver utilize a banda de diversas antenas adjacentes ao cliente para o suprimento da banda, chamado de
    modelo parcial,
    a segunda definição define que apenas uma antena forneça toda a banda exigida pelo cliente, chamado de modelo total.
    Como esses modelos ja foram definidos em relatórios e em aulas passadas, não definirei as restrições aqui novamente.

    \section{Modelos}

    Uma particularidade dos meus dois modelos é que o cálculo de adjacência não é feito dentro do solver.
    Utilizo os seguintes passos para definir a restrição de adjacência nos dois modelos:

    \begin{algorithm}[H]
        \For{every client $i$}{
        \For{every antenna $j$}{
        \If{i is adjacent to j}{
        $usage$[$i$][$j$] = 0;
        }\EndIf
        }\EndFor
        }\EndFor
    \end{algorithm}

    Lembrando que $usage$ é e uma matriz de variáveis de decisão que relaciona clientes e antenas, e o valor armazenado dentro dela é a quantidade de banda que um cliente
    $i$ está usando da antena $j$ no modelo parcial, no modelo total essa matriz é binária, e o valor dentro dela é $1$ se a antena $j$ atende o cliente $i$.
    Como é possível ver, para cada cliente, eu itero por cada antena, e se os dois não forem adjacentes eu "forço" o modelo a utilizar 0 na relação.
    Essa operação é custosa, pois além de eu realizar $N$*$M$ iterações (sendo $N$ o número de clientes e $M$ o número de antenas), o calculo de
    adjacência é uma operação lenta, isso adiciona um atraso no meu programa.


    \section{Resultados}
    Vale lembrar que estou rodando um script em Python e faz as chamadas no solver CPLEX utilizando as libs em C.

    \begin{enumerate}
        \item Solve time é o tempo que o solver levou para resolver o problema.
        \item Execution time é o tempo que o programa demorou para executar completamente.
    \end{enumerate}

    \begin{center}
        \begin{tabular}{ | l | l | l | p{5cm} |}
            \hline
             & Modelo Parcial & Modelo Total \\ \hline
            Ticks & 129.47 & 234.88 \\
            Solve time & 0.31 sec & 0.67 sec \\
            Execution time & 10.546 sec & 12.551 sec \\
            \hline
        \end{tabular}
    \end{center}

    \section{Conclusão}
    Como é possível ver, o modelo parcial conseguiu resolver o problema mais rápidamente, acredito que o motivo seja porque ele trabalha
    com as variáveis reais da matriz $usage$, enquanto que no modelo total, a matrix $usage$ é uma variável binária.
    É possível ver que o programa levou mais tempo definindo as restrições do que resolvendo o modelo em sí.


\end{document}